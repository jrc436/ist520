\section{Page Numbering, Headers and Footers}
Your final submission SHOULD NOT contain any footer or header string information 
at the top or bottom of each page. The submissions will be paginated in a determined 
order by the chairs and page numbers added to the pdf during the compiling, 
indexing, and pagination processes.

\section{Producing and Testing PDF Files}

We recommend that you produce a PDF version of your submission well
before the final deadline.  Your PDF file must be ACM DL
Compliant. The requirements for an ACM Compliant PDF are available at:
{\url{http://www.sheridanprinting.com/typedept/ACM-distilling-settings.htm}}.

Test your PDF file by viewing or printing it with the same software we
will use when we receive it, Adobe Acrobat Reader Version 7. This is
widely available at no cost from~\cite{acrobat}.  Note that most
reviewers will use a North American/European version of Acrobat
reader, which cannot handle documents containing non-North American or
non-European fonts (e.g. Asian fonts).  Please therefore do not use
Asian fonts, and verify this by testing with a North American/European
Acrobat reader (obtainable as above). Something as minor as including
a space or punctuation character in a two-byte font can render a file
unreadable.

\section{Blind Review}

For archival submissions, CHI requires a ``blind review.'' To prepare
your submission for blind review, remove author and institutional
identities in the title and header areas of the paper. You may also
need to remove part or all of the Acknowledgments text.  Further
suppression of identity in the body of the paper and references is
left to the authors' discretion. For more details, see the submission
guidelines and checklist for your submission category.

\section{Conclusion}

It is important that you write for the SIGCHI audience.  Please read
previous years' Proceedings to understand the writing style and
conventions that successful authors have used.  It is particularly
important that you state clearly what you have done, not merely what
you plan to do, and explain how your work is different from previously
published work, i.e., what is the unique contribution that your work
makes to the field?  Please consider what the reader will learn from
your submission, and how they will find your work useful.  If you
write with these questions in mind, your work is more likely to be
successful, both in being accepted into the Conference, and in
influencing the work of our field.

\section{Acknowledgments}

We thank CHI, PDC and CSCW volunteers, and all publications support
and staff, who wrote and provided helpful comments on previous
versions of this document.  Some of the references cited in this paper
are included for illustrative purposes only.  \textbf{Don't forget
to acknowledge funding sources as well}, so you don't wind up
having to correct it later.