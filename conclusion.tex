\section{Conclusion}
In this review, we have covered the evolving viewpoint of the theory of Collective Intelligence and its applications. The Collective Intelligence theory began as a methodology to study classical information processing, modeling a group of people as a distributed computer. It later became a way to discuss a method of measurement for general intelligence of a group of people, in a similar way to how the IQ test formalized this idea for a single person. Later, it became a method for analyzing intelligence at scale: how to get good results out of a large number of people performing a task. We have went through two applications, mostly viewing CI in this latter sense, the WikiCrimes system and a rating system. 

There are some criticisms of the theory from two standpoints. From a theoretical standpoint, it does not seem to say very much that distributed cognition does not, and it's unclear what exactly crowdsourcing is, depending on who's talking. From an application standpoint, the theory suffers from ethical dilemmas, is limited in scope, and fails to consider the full capability of humans.

These criticisms are only partially valid on the actual applications we presented. Both to an extent involve relatively complicated tasks, and especially with the WikiCrimes system, possible benefits to the users. In a sense, this is because the WikiCrimes system is a collaborative crowdsourcing platform instead of a directed crowdsourcing platform. However, we still contend neither of these applications make use of people's ability to learn, which is fundamental to human intelligence. 

With the rise of MOOCs in popularity and the massive amount of unlabeled data that can now be collected, Collective Intelligence is in a good position for application in the future in the fields of CSCW and HCI.