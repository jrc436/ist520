\section{Introduction}

Although the term Collective intelligence is not used until recent years, activities behind this thought, for example, Condorcet's jury theorem~\cite{ landemore2012democratic}, have existed for a long time. Generally, collective intelligence indicates that intelligence is distributed across individuals and needs to be collected, coordinated and crystallized for consensus decision making through collaboration.

Since John Smith introduced classic collective intelligence in 1990s~\cite{cibook}, the theory of collective intelligence has being developed for decades. In classic collective intelligence, collaboration is considered as information processing activity. Computer system, as well as human, is a part for constituting collaboration system. In recent years, collective Intelligence has become a widely used theory for analyzing the performance of groups, either large or small-scale. 

MOOC (Massive Open Online Courses) is an example of collective intelligence in large scale. In MOOC, students collaboratively build knowledge "around the course content and for self reflection"~\cite{de2012merging}.
CIR (Citizen's Initiative Review) is an example of collective intelligence in small group. In CIR, some representatives review ballot initiatives and
candidates on behalf of the general public to generate statements that can help government to make decisions~\cite{atlee2008co}.

In the rest of this paper, we firstly introduce the concepts of collective intelligence, especially how it can be associated with computer. Then we concentrate on crowdsourcing tasks, which often serves as the context of collective intelligence, including directed crowdsourcing, collaborative crowdsourcing and passive crowdsourcing. Small-Scale case is considered as well. Then the limitations of collective intelligence, for example, scope, applicability and potential ethical dilemmas are discussed. After that, we describe two applications of collective intelligence. One is an open innovation and crowdsourcing for commercial ideas; the other is to build a collaborative crime report system. In both cases, our focus is on how to apply collective intelligence theory to real situation. In theory police section we revisit collective intelligence and compare it with distributed cognition. We conclude the paper with some findings at last.
