\section{Small-scale CI}
Similar to \lq general intelligence\rq ~\cite{deary2000looking} which indicates the charactieristic levels of intelligence for individuals, researchers ~\cite{science} have studied and shown that there exists a single statistical factor \lq c factor\rq : the \lq collective intelligence\rq that could explain the performance of small groups (e.g. two to five people) on a wide range of tasks. A group's collective intelligence infers that a group's ability to perform one task is correlated with its ability to perform a wide range of other tasks ~\cite{science}. Empirically, the average inter-item correlation for group scores on different tasks is positive, and there is a factor extracted in the factor analysis of these scores which accounts for 30$\%$ to 50$\%$ of the variance, while the following factors account for substantially less variance ~\cite{science}. 

In ~\cite{science}, the authors first support the existance of such a general collective intelligence factor by demonstrating their results with two studies. The first study envolves 40 three-person groups working on tasks from the McGrath Task Circumplex ~\cite{mcgrath1984groups} and the results positively support the authors' hypothesis. The collective intelligence factor also has significant effect when predicting the performance on the criterion task, while the average individual intelligence and the maximum individual intelligence do not. In the second study, the authors use 152 groups ranging from 2 to 5 peopleand 5 additional tasks in order to replicate their findings in groups of different sizes and a broader sample of tasks. As expected, the results again supported the existance of a general collective intelligence factor.

The authors in ~\cite{science} also find out three main facots that cause the collective intelligence factor. First, collective intelligence is significantly correlated with the average social sensitivity of group members. Second, groups where the the conversation is dominated by fewer peope has less collective intelligence. Third, their results show that the c factor is positively and significantly correlated with the proportion of females group members. However, this find largely overlaps with the first finding as women in their study have better score on social sensitivity.

The findings in ~\cite{science} show that collective intelligence depend both on the properties of the group (such as the average intelligence of group members) and on how the groups members interact with each other. These findings open the door to many important research questions to raise the intelligence of a group as a whole rather than on the individual scale. 
